We have performed experiments to evaluate the accuracy of Prototypical Networks, Wide-DenseNet, MAML and Transfer Learning on three handshape recognition datasets. For all datasets we found models that showed a performance on par with or better than the state of the art. All models achieve near-perfect accuracy on CIARP. This shows that the dataset is too simple as a benchmark for handshape recognition. While it has more samples than the other datasets (6000), they are too homogeneous and do not have enough variation.

In  future  work,  we  will  focus  on  comparing  with  other  datasets  to  better understand the relationship between models and dataset complexities for hand-shape  recognition.  We  also  see  the  need to  compare with the use of MAML models pretrained with different tasks, combining datasets to achieve it. Adding to future plans, we are going to research on methods that can take advantage of unlabeled data and investigate the possibility of merging data sets from different sign languages to augment the sample size, as well as identify the types of data augmentation that lead to an improvement in state-of-the-art models.
